% Les deux résumés doivent tenir sur une page
\begin{resumefr}
Durant ce rapport, nous avons commencé par nous familiariser avec les images obtenues et la notion de magnitude, pour ensuite calibrer nos observations en détectant les étoiles sur notre image. La détection stellaire dépend du rayon d'ouverture que l'on prend pour chaque étoile. Avec plusieurs méthodes, nous avons déterminé la méthode optimale pour obtenir le rayon optimale d'ouverture pour chaque étoile. En obtenant grâce à Vizier, la magnitude des étoiles les plus lumineuses, nous avons pu déterminer la magnitude des étoiles de plus faible luminosité. 
\end{resumefr}

\vskip5.em

\begin{resumeen}


During this report, we began by familiarising ourselves with the images obtained and the concept of magnitude, then calibrated our observations by detecting the stars in our image. Star detection depends on the aperture radius used for each star. Using several methods, we determined the optimal method for obtaining the optimal aperture radius for each star. By using Vizier to obtain the magnitude of the brightest stars, we were able to determine the magnitude of the dimmer stars.

\end{resumeen}

\vskip5.em

\begin{contribution}
 Lorys s'est occupé de la partie introductive du rapport, de la moitié du code ainsi que de ses commentaire. Silio s'est occupé des graphes, de l'autre moitié du code ainsi que du la partie calibration du rapport.
\end{contribution}

\vskip5.em

\begin{remerciements}
Merci à M.Morin pour ses précieux conseils durant tout ce projet.
\end{remerciements}


