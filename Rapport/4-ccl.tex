\section{Conclusion}
% Résumé bref des étapes de traitement
% Description brève des mesures photométriques finales
% Brève analyse astrophysique, commentaires pertinents, tâches spécifiques à chaque projet






La calibration de la magnitude nous permettant d'obtenir la constante B, celle-ci nous donne une bonne indication sur la précision de la calibration photométrique de nos valeurs. Notre meilleure valeur de B, ayant une incertitude de 0.062, notre calibration est relativement précise. Pour aller plus loin, il serait intéressant de développer une méthode robuste pour la séparation des étoiles, problématique majeure de notre méthode.