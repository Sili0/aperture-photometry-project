\section{Description des observations spectroscopiques}
%

\subsection{Instrumentation}
% Brève description de l'instrumentation utilisée : télescope, spectrographe, caméra CCD

\subsection{Observations}
\subsubsection{Objets astrophysiques}
% Brève description des observations astronomiques réalisées : pour le ou les objets cibles inclure :
%  * un diagramme d'observabilité
%  * une carte de champ large en indiquant l'étoile observée sur les cartes téléchargées à https://www.iau.org/public/themes/constellations/
%  * une carte de champ resserrée produite avec Aladin ou pyastroplan et couvrant un champ de 2°x2°
%  * un bref commentaire sur l'utilité de ces graphiques
% * Un journal des observations : combien de spectres, temps de pose, Nb ADU sous forme de description et/ou de tableau

\subsubsection{Images de calibration}
% Description détaillée des différentes images acquises 

\subsubsection{Caractérisation de la caméra CCD}
% Caracterisation de la caméra CCD : gain (e-/ADU), bruit de lecture (e-), courant d'obscurité % (e-/s/pix)
% Expliquer comment ces grandeurs sont déterminées et comparer aux données constructeur
% Expliquer brièvement à quoi correspondent ces grandeurs
La caractéristation du CCD est importante car elle permet de donner toutes les méta données importantes à nos images. 
Toutes ces données se trouvent au niveau du Header d'une image. Notre CCD est donc caractériser par les données suivantes :
